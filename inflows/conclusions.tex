\documentclass{article}
\usepackage{fullpage}
\newcommand{\rvir}{R$_{vir}$}


\title{Conclusions from Inflow Analysis}

\begin{document}
\maketitle



\section{Overview}

In vela2b, halo number 27, there is a structure with abnormally low
metallicity.  It is a column-like structure that extends from the edge of the 
box to the galaxy itself. The gas is generally infalling and cold, with  
characteristic temperatures around $10^4$ K and ddensities ranging from 
$10^{-2}\,-\,10^{-6}$.


To explore the structure, we explored the time evolution of several parameters
consisting of various spatial and kinematic structures, as well as metallicity.
We also looked at larger temperature ranges. These are broken up into ``cool''
gas ($3.5\,<\,\log(T)\,<\,4.5$), ``warm'' gas ($3.5\,<\,\log(T)\,<\,4.5$), and
``hot'' gas ($3.5\,<\,\log(T)\,<\,4.5$). A very import conlclusion is that
within a given temperature range, the density range selected corresponds to a
galectocentric distance, with lower densities corresponding to a larger
distance. For a given density range, as the temperature increases the
corresponding distance decreases. For our cool gas, at $z\,=\,0$, the density
ranges we chose correspond to distances ranging from 0.5 R$_{vir}$ to 4.0
R$_{vir}$, while hot gas in the same redshifts only range from 0.5 R$_{vir}$
to 1.5 R$_{vir}$, implying that hot gas is mostly contrained to the CGM. 



\section{Spatial}
Over time the filament's spatial distribution changes, but always occupies the same 
region of space. At high redhsift the filament is not well constructed. At $z\,=\,4$
the filament consists of several small ($\sim$ 10 kpc) clouds that do not form a
coherent structure. The clouds quickly come together as time passes. As time
passes, the filament becomes more coherent and structured. It eventualy extends
completely from the galaxy to the edge of the box. 

\subsection{Size}
When looking at the cool gas, the type of gas selected by differing density cuts
fills the space differently and changes over time. 
\begin{itemize}
\item {\bf Highest Density}: $-3.0\,<\,\log(n_H)\,<\,-2.5$. Contains most of the
mass before the merger starts. Once the merger ends, this phase contains at most
10\% of the mass in the filament. This structure's size stays at a fairly
constant 50 pkpc. 
\item {\bf High Density}: $-3.5\,<\,\log(n_H)\,<\,-3.0$. This phase contains a
roughly constant 20-25\% of the filament's mass, but it does drop with time. By
z=1, it only contains 5\% of the mass. The amount of mass in this phase increases
during outflows of the starburst. It's size is roughly constant in shape of
$\sim$ 75 pkpc. 
\item {\bf High-mid Density}: $-4.0\,<\,\log(n_H)\,<\,-3.5$. The amount of mass
in this phase starts low and increases with time, increasing from $\sim$ 7\% to a
peak of 40\% at a=0.43. It then drops after the peak to less than 20\%. It's size
is roughly constant at roughly 100 pkpc. It peaks at the beginning of the merger
at a value of 125 pkpc, then settles at 100 pkpc after the merger before dropping
at a=0.43 to 50 pkpc. 
\item {\bf Low-mid Density}: $-4.5\,<\,\log(n_H)\,<\,-4.0$. This phase doesn't
really exit before the merger, containly less than 5\% of the mass. Once the
merger starts, the amount of mass in this phase rises rapidly until it reaches a
peak of $\sim$ 25\% during the starburst. After this, it drops to $\sim$ 15\% at
a=0.40. It then rises rapidly to roughly 40\% of the mass. It's physical size is
not constant with time. It starts small (75 pkpc), rises to 150 pkpc during the
merger, before shrinking ot 75 pkpc. The timing of the peak changes based on the
mass fraction used. For 25\% of the mass of this phase, the structure is only 100
pkpc in size and peaks at a=0.25. 
\item {\bf Low Density}: $-5.0\,<\,\log(n_H)\,<\,-4.5$. Before the merger, this
phase has less than 2\% of them mass. It doesn't become relavent until the merger
starts. The amount of mass in this phase increases during the merger and
starburst until it peaks at $\sim$ 17\%. It then drops until a = 0.43 (5\%), then
increases again until it reaches an all-time high of 25\%. Physically, this phase
is much bigger. It starts with a size of 75 pkpc and increases intil a peak of
225 pkpc at a=0.40. Oddly, the peak of the size depends on what mass fraction the
radius includes. For 25\% of the mass in this phase, the structure reaches its
maximum of 110 kpc at a=0.29. 
\item {\bf Lowest Density}: $-5.5\,<\,\log(n_H)\,<\,-5.0$. This phase is by far
the lease important to the mass budget as it peaks at 3\% of the mass at a=0.40.
It follows the same trend with time as the low density gas. This phase is also
the most extended, starting at 75 pkc and peaking at $>$200 pkpc. There is less
trends with time with this phase, possibly due to its low mass. This implies few
cells exist in this phase (as expected being so close to the edge of the phase
diagram) and small shifts create stochasitc behaviours. 
\end{itemize}

Additional, the importance of the galaxy can be determined by examining the size
of the filament in units of R$_{vir}$. Depending on the density criteria and the
time, $R_{90}$, the radius of the filament that contains 90\% of the filament's
mass in that density range, can range from 0.5 to 3 \rvir.  For all density
selections, $R_{90}$ decreases with time. This is expected as \rvir grows with
time and the three densest bins are roughly constant in size. However even the
lower density bins cannot make the filament grow fast enough to outpace \rvir. As
time goes on, the galaxy itself become more relavant as it outgrows the filament.
If this simulation were continued to lower redshifts, it would seem that
eventually the filament would be insignificant to the evolution of the galaxy.
Thus filament behaviours can only really influence the galaxy's evolution at high
redshift. 


\subsection{Location}
The mean density of the gas is highly correlated with the distance it lies from
the galaxy, with lower density gas residing farther from the center. This trend
exists for all time and phases. Hotter gas tends to reside closer to the galaxy
as well. For the ``cool'' gas, changing the density selection can probe
galactocentric distances from within the CGM to 4.5 R$_{vir}$ away. However the
``hot'' gas only extends out to 1.5 R$_{vir}$ at z=1. Once the filament is formed
(starting at a=0.25), the mean distance of each density cut do not cross, meaning
this trend holds during the merger and starburst. 

The box coordinates of the filament do not change with time. Once the filament
forms, it is constant in space. The galaxy itself roles around, changin
orientation often. This is excaserbated by the merger. However, the filament
itself does not move, implying that the filament is not a huge source of angular
momentum to the galaxy. This could be due to the filament being much bigger than
the galaxy and roughly the size of its halo. 


\section{Kinematics}



\section{Metallicity}






















\end{document}












