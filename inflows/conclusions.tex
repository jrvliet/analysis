\documentclass{article}
\usepackage{fullpage}
\usepackage{graphicx}
\usepackage{subfig}
\usepackage{float}

\newcommand{\rvir}{R$_{vir}$}
\newcommand{\cool}{``cool''}
\newcommand{\warm}{``warm''}
\newcommand{\hot}{``hot''}
\newcommand{\kms}{km s$^{-1}$}


\title{Conclusions from Inflow Analysis}

\begin{document}
\maketitle



\section{Overview}

In vela2b, halo number 27, there is a structure with abnormally low
metallicity.  It is a column-like structure that extends from the edge of the 
box to the galaxy itself. The gas is generally infalling and cold, with  
characteristic temperatures around $10^4$ K and densities ranging from 
$10^{-2}\,-\,10^{-6}$. The mean density of the gas decreases over time, from
$10^{-3}$ to $10^{-4}$ cm$^{-3}$. This is shown in Figure \ref{generalTrend}.

\begin{figure}[h]
\centering
\includegraphics[width=0.95\textwidth]{./filamentProps.png}
\caption{Evolution of the general properties of the filament. The left panel
shows the mass contained within the filament. The middle and right panels show
the evolution of the density of the filament, with the mean in the middle and
median on the right.}
\label{generalTrend}
\end{figure}


To explore the structure, we explored the time evolution of several parameters
consisting of various spatial and kinematic structures, as well as metallicity.
We also looked at larger temperature ranges. These are broken up into ``cool''
gas ($3.5\,<\,\log(T)\,<\,4.5$), ``warm'' gas ($3.5\,<\,\log(T)\,<\,4.5$), and
``hot'' gas ($3.5\,<\,\log(T)\,<\,4.5$). A very import conlclusion is that
within a given temperature range, the density range selected corresponds to a
galectocentric distance, with lower densities corresponding to a larger
distance. For a given density range, as the temperature increases the
corresponding distance decreases. For our cool gas, at $z\,=\,0$, the density
ranges we chose correspond to distances ranging from 0.5 R$_{vir}$ to 4.0
R$_{vir}$, while hot gas in the same redshifts only range from 0.5 R$_{vir}$
to 1.5 R$_{vir}$, implying that hot gas is mostly contrained to the CGM. 



\section{Spatial}
Over time the filament's spatial distribution changes, but always occupies the same 
region of space. At high redhsift the filament is not well constructed. At $z\,=\,4$
the filament consists of several small ($\sim$ 10 kpc) clouds that do not form a
coherent structure. The clouds quickly come together as time passes. As time
passes, the filament becomes more coherent and structured. It eventualy extends
completely from the galaxy to the edge of the box. 

\subsection{Size}
When looking at the cool gas, the type of gas selected by differing density cuts
fills the space differently and changes over time. See Figure \ref{massInside}
for the following points. 

\begin{figure}[h]
\centering
\includegraphics[width=0.95\textwidth]{./massContained_denseCuts.png}
\caption{Evolution of the size of the filament. The upper right shows the amount
of mass in each phase cut as a fraction of the total mass in the filament. The
upper left shows the radius from the center of the filament that encloses 90\% of
the mass of the filament in that phase cut, as measured in \rvir. The bottom
lefts shows the same quantity but in units of proper kpc. The bottom right shows
the same but in units of comoving kpc.}
\label{massInside}
\end{figure}


\begin{itemize}
\item {\bf Highest Density}: $-3.0\,<\,\log(n_H)\,<\,-2.5$. Contains most of the
mass before the merger starts. Once the merger ends, this phase contains at most
10\% of the mass in the filament. This structure's size stays at a fairly
constant 50 pkpc. 
\item {\bf High Density}: $-3.5\,<\,\log(n_H)\,<\,-3.0$. This phase contains a
roughly constant 20-25\% of the filament's mass, but it does drop with time. By
z=1, it only contains 5\% of the mass. The amount of mass in this phase increases
during outflows of the starburst. It's size is roughly constant in shape of
$\sim$ 75 pkpc. 
\item {\bf High-mid Density}: $-4.0\,<\,\log(n_H)\,<\,-3.5$. The amount of mass
in this phase starts low and increases with time, increasing from $\sim$ 7\% to a
peak of 40\% at a=0.43. It then drops after the peak to less than 20\%. It's size
is roughly constant at roughly 100 pkpc. It peaks at the beginning of the merger
at a value of 125 pkpc, then settles at 100 pkpc after the merger before dropping
at a=0.43 to 50 pkpc. 
\item {\bf Low-mid Density}: $-4.5\,<\,\log(n_H)\,<\,-4.0$. This phase doesn't
really exit before the merger, containly less than 5\% of the mass. Once the
merger starts, the amount of mass in this phase rises rapidly until it reaches a
peak of $\sim$ 25\% during the starburst. After this, it drops to $\sim$ 15\% at
a=0.40. It then rises rapidly to roughly 40\% of the mass. It's physical size is
not constant with time. It starts small (75 pkpc), rises to 150 pkpc during the
merger, before shrinking ot 75 pkpc. The timing of the peak changes based on the
mass fraction used. For 25\% of the mass of this phase, the structure is only 100
pkpc in size and peaks at a=0.25. 
\item {\bf Low Density}: $-5.0\,<\,\log(n_H)\,<\,-4.5$. Before the merger, this
phase has less than 2\% of them mass. It doesn't become relavent until the merger
starts. The amount of mass in this phase increases during the merger and
starburst until it peaks at $\sim$ 17\%. It then drops until a = 0.43 (5\%), then
increases again until it reaches an all-time high of 25\%. Physically, this phase
is much bigger. It starts with a size of 75 pkpc and increases intil a peak of
225 pkpc at a=0.40. Oddly, the peak of the size depends on what mass fraction the
radius includes. For 25\% of the mass in this phase, the structure reaches its
maximum of 110 kpc at a=0.29. 
\item {\bf Lowest Density}: $-5.5\,<\,\log(n_H)\,<\,-5.0$. This phase is by far
the lease important to the mass budget as it peaks at 3\% of the mass at a=0.40.
It follows the same trend with time as the low density gas. This phase is also
the most extended, starting at 75 pkc and peaking at $>$200 pkpc. There is less
trends with time with this phase, possibly due to its low mass. This implies few
cells exist in this phase (as expected being so close to the edge of the phase
diagram) and small shifts create stochasitc behaviours. 
\end{itemize}

Additional, the importance of the galaxy can be determined by examining the size
of the filament in units of R$_{vir}$. Depending on the density criteria and the
time, $R_{90}$, the radius of the filament that contains 90\% of the filament's
mass in that density range, can range from 0.5 to 3 \rvir.  For all density
selections, $R_{90}$ decreases with time. This is expected as \rvir grows with
time and the three densest bins are roughly constant in size. However even the
lower density bins cannot make the filament grow fast enough to outpace \rvir. As
time goes on, the galaxy itself become more relavant as it outgrows the filament.
If this simulation were continued to lower redshifts, it would seem that
eventually the filament would be insignificant to the evolution of the galaxy.
Thus filament behaviours can only really influence the galaxy's evolution at high
redshift. 


\subsection{Location}
The mean density of the gas is highly correlated with the distance it lies from
the galaxy, with lower density gas residing farther from the center. This trend
exists for all time and phases. Hotter gas tends to reside closer to the galaxy
as well. For the ``cool'' gas, changing the density selection can probe
galactocentric distances from within the CGM to 4.5 R$_{vir}$ away. However the
``hot'' gas only extends out to 1.5 R$_{vir}$ at z=1. Once the filament is formed
(starting at a=0.25), the mean distance of each density cut do not cross, meaning
this trend holds during the merger and starburst. These correlations are shown in
Figure \ref{rModmean}.

\begin{figure}[h!]
\centering
\includegraphics[width=0.95\textwidth]{./denseCuts/simple/density_cuts_evolution_snIImean.png}
\caption{Evolution of the distance from the galaxy for each phase selection,
measured in units of \rvir.}
\label{rModmean}
\end{figure}

The box coordinates of the filament do not change with time. Once the filament
forms, it is constant in space. The galaxy itself roles around, changin
orientation often. This is excaserbated by the merger. However, the filament
itself does not move, implying that the filament is not a huge source of angular
momentum to the galaxy. This could be due to the filament being much bigger than
the galaxy and roughly the size of its halo. 


\section{Kinematics}
The kinematics of the gas is more dependent on the temperature selection than the
density selection. 

\subsection{Speed}
As expected, the cooler gas moves slower than the hotter gas,
with typical speeds of 50--150 km/s for \cool gas to 200--350 km/s for \hot
gas. The \cool and \warm gas share similar speed evolutions, starting at low
speeds and increasing until the merger ends. Both phases have the highest speed
during at the end of the merger, around 250 km/s. They quickly slow down and
settle in to steady-state speeds. Only after the merger do the different density
cuts start to seperate out, with the dense gas having lower speeds ($\sim$50
km/s) and the diffuse gas having higher speeds ($\sim$150 km/s). There is very
little spread in the speed as probed by each density cut, with
$\sigma_{cool}\,<\,50$ km/s and $\sigma_{warm}\,<\,75$ km/s. See Figure
\ref{speedMean}.

\begin{figure}[h!]
\centering
\includegraphics[width=0.95\textwidth]{./denseCuts/simple/density_cuts_evolution_speedMean.png}
\caption{Evolution of the speed of the filament gas}
\label{speedMean}
\end{figure}

The \hot gas
does not have any strong evolution nor any density trends. The \hot gas speed is
not affected by the merger nor the starburst, interesting because this gas traces
the outflows generated by the starburst as discussed in the next section. 
$\sigma_{hot}\,\sim\,125$ km/s.


\subsection{Radial Velocity}
Perhapse the strongest difference between the different temperature selections is
the radial velocity of the gas. The \cool gas is consistently infalling, with
$\bar{v_{r}}\,<\,0$ for all density selections and across all snapshots. It falls
with the greatest velocity during the merger, where all the \cool gas moves
towards the galaxy with speeds greater than 100 km/s. Even outside of the merger,
the \cool gas is generally infalling with speeds of $\sim$ 50 \kms. There is no
trend with density for the \cool gas. These and following trends are shown in
Figure \ref{vrmean}.


\begin{figure}[h!]
\centering
\includegraphics[width=0.95\textwidth]{./denseCuts/simple/density_cuts_evolution_vrMean.png}
\caption{Evolution of the mean radial velocity of the filament gas.}
\label{vrmean}
\end{figure}

The \warm gas is, on average, not infalling nor outflowing. During the merger, it
traces infalling material, moving as fast as the \cool gas is. However outside of
the merger, $\bar{v_{r,warm}}\,\sim\,0$ \kms. There is also no trend with density
for the \warm gas. 

The \hot gas is mostly outflowing, with positive $v_r$ for nearly all snapshots.
During the merger, some of the \hot gas is inflowing, but never very quickly.
This phase of gas is very sensitive to the starburst, which quickly changes the
inflowing gas to outflowing. Once the starburst begins, the \hot gas moves
outwards with roughly constant speeds of 100-250 \kms. Unlike the \cool and \warm
gas, there is a strong density trend in the \hot gas, where the denser gas moves
slower outwards than the diffuse gas. 

The sensitivity of the \hot gas could be an artifact of its spatial location.
By the time the starburst starts, most of the \hot gas resides within 1.5 \rvir
of the galaxy, inside the classically defined CGM. This proximity is going to
make it easier for it to be affected by winds from the galaxy. Meanwhile, the
\cool gas mostly resides far outside the CGM and would be much more difficult for
it to be affected by the winds. The \cool gas also has a small $R_{90}$, meaning
it resides closer to the core of the filament. The higher temperature gas is
farther outside of the filament. This structure shields the denser, inner gas
from the winds. Thus the \hot gas being affected most by the winds is not
surprising for multiple reasons. 




\section{Metallicity}
The metallicity fo the filament is the most puzzeling. In general, the gas
contained in the filament is very metal poor (this is how it was initially
found). However, all selection for gas in the filament is now based on density
and temperature. The metallicity of the gas is measured by SNII and SNIa mass
fraactions. These are the fraction of the total mass in each cell that is due to
metals from type II supernovae and type Ia supernovae, respectively. Using either
type of supernovae as the tracer yields the same results. 

In general, the metallicity of the filament increases with time, as expected. The
increase is slight, however, increasing by 1 dex at most. This trend and the
following are shown in Figure \ref{snIImean}.

\begin{figure}[h!]
\centering
\includegraphics[width=0.95\textwidth]{./denseCuts/simple/density_cuts_evolution_snIImean.png}
\caption{Evolution of the metallicty of the filament gas as traced by the mean
SNII mass fraction}
\label{snIImean}
\end{figure}

The mean metallicity of the \cool gas varies wildly based on the density
selection. Dense gas, close to the galaxy, increases in metallicity from
$10^{-3.5}$ to $10^{-2.5}$. As the selected density decreases, the metallicity
decreases and the slope over time decreases. The most diffuse gas, which resides
the farthest from the galaxy, is very metal poor ($\sim\,10^{-4.5}$) and shows no
evolution with time. This gas is ``pristine'' cosmic gas and is not affected by
the star formation within the galaxy. 

The \warm gas has a much smaller density dependence. There is only a spread due
to density selection after the starburst. However this gas shows the most
puzzling feature found. All the \warm gas has similar metallicity with no
evolution ($\sim\,10^{-3.3}$) until the merger starts. Once the merger ends, and
the starburst starts, the metallicity of the dense gas (close to galaxy) starts
to increase. Its slope ($\frac{dZ}{dt}$) is fairly constant, even after the
starburst ends. The lowest denstiy gas actually drops in metallicty during the
starburst, down to almost $10^{-4}$. Once the starburst stops, this diffuse
material starts to increase in metallicity. The time between when the metallicity
drops and when it starts to increase changes with the density cut. It occurs
later with more diffuse material. This material is farther from the galaxy and
thus takes longer to react to the metal rich outflows from the galaxy. This delay
is evidence of winds from the starburst. Once the winds reach the diffuse
material, its metallicity starts to increase with comparable slopes to the dense
gas. 

There is a puzzle though. Why does the metallicity in the diffuse warm gas 
{\it drop} when the merger ends? A drop in metallicity suggests an infusion of
pristine material, but where does this come from? Is the merging galaxy bringing
in its own filamentary bubble? Are the metals being pushed out of the warm gas?
This is an open question. Maybe the initial winds from the starburst first push
out the metal poor halo around the host and the incoming satellite, meaning winds
are formed from two pieces: and initial, brief, metal-poor gust and a secondary, 
sustained, metal-rich wind. This inplies that the winds do not mix with the
filament material. Ths winds just push it away, at least initially. 

The \hot gas has very little variation across densities, likely due to small
variation in galactocentric distance between the differing cuts. There is a small
trend of increasing metallicity with increasing density. There is a similar wind
propigation affect as seen in the \warm gas, but it is smaller. Likewise, there
is a similar decrease in metallicity in the diffuse hot gas as to what is seen in
the diffuse warm gas. 







\section{Summary}

The key points are:
\begin{itemize}
\item Vela2b-27 has a prominant filament structure of cool, inflowing gas
\item The filament takes time to form
\item Once formed, the filament is structurally stable against major mergers and
starformation activity within the galaxy
\item Temperature of the filament is constant in time
\item Denisty of the filament decreases with time
\item Observing various phases of the filament probes different parts of it that
can have very different spatial and kinematic strucures
\item The filament becomes less important in driving the galaxy's evolution with
time
\end{itemize}
















\end{document}












